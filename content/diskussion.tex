\section{Diskussion}
\label{sec:Diskussion}
In dem Graphen zur Energieverteilung sieht man für $\SI{20}{\degreeCelsius}$ gut, dass die Verteilung mittels der Fermi-Dirac-Statisik geschieht.
Für $\SI{140}{\degreeCelsius}$ ist die Kurve wie für höhere Temperatur zu erwarten stark abgeflacht und am oberen Ende abgeschnitten.
Aufgrund der erhöhten Temperatur kommt es zu mehr Teilchenbewegung im Gas.
Die Anzahl der inelastischen Stöße nimmt zu.
Es kommt zusätzlich vermehrt zu Stößen, welche nicht nur in der $z$-Richtung des Feldes stattfinden.
Diese Elektronen werden nicht detektiert.
An der Franck-Hertz-Kurve ist klar erkennbar, dass die Kurve besonders für hohe Beschleunigungspannungen stark abgeflacht ist, was auf die elastischen Stöße zurückgeführt werden kann.
Die Kurve für die Ionisationsenergie hat nur wenig Ähnlichkeit zur erwarteten Kurve, jedoch liegt der erhaltenen Wert in der richtigen Größenordung für Ionisationsspanunngen.
Der Wert besitzt trotzdem noch eine große Abweichung vom tatsächlichen Wert von $\SI{10.438}{\electronvolt}$\cite{hg}.
