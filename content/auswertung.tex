\section{Auswertung}
\label{sec:Auswertung}
\subsection{Bestimmung der Energieverteilung der Elektronen}
Die Messwerte zur Bestimmung der Energieverteilung bei $\SI{20}{\degreeCelsius}$ sind in Tabelle \ref{tab:e20} aufgetragen.
Die Messwerte werden dem Graphen des $XY$-Schreibers so entnommen, dass für große Steigung viele Werte aufgenommen werden.
Die Höhe ist hierbei eine Größe proportional zum Strom $I_A$.
\begin{table}[H]
    \caption{Messwerte der Energiekurve bei $\SI{20}{\degreeCelsius}$.}
    \label{tab:e20}
    \centering
    \begin{tabular}{S[table-format=2.3(0)e0] S[table-format=2.2(0)e0]}
        \toprule
{$U/\si{\volt}$} & { Höhe$/\si{\centi\meter}$} \\
		\midrule
1.000 &      14.65 \\
2.000 &      14.60 \\
3.000 &      14.50 \\
4.000 &      14.40 \\
5.000 &      14.20 \\
6.000 &      14.00 \\
6.500 &    13.85 \\
7.000 &      13.60 \\
7.500 &    13.30 \\
8.000 &      12.95 \\
8.500 &    12.40 \\
9.000 &      11.55 \\
9.250 &   10.95 \\
9.500 &    10.00 \\
9.625 &  8.70 \\
9.750 &   6.30 \\
9.813 &  4.5 \\
9.875 &  2.70 \\
10.000 &      1.00 \\
        \bottomrule
    \end{tabular}
\end{table}

\noindent Die differentielle Energieverteilung, welche sich ergibt aus $I_A(U_A)-I_A(U_A+\symup\Delta U_A)$ aufgetragen gegen $U_A$ aufgetragen, ist in Diagramm \ref{fig:e20} dargestellt.
\begin{figure}[H]
	\centering
	\includegraphics[width=0.8\textwidth]{build/energie20.pdf}
	\caption{Differentielle Energieverteilung der Elektronen bei $\SI{20}{\degreeCelsius}$.}
	\label{fig:e20}
\end{figure}

\noindent Aus dem Diagramm \ref{fig:e20} lässt sich nun über das Maximum das Kontaktpotential bestimmen:
\begin{equation}
	K = U_B - {U_B}_\text{eff} = \SI{1.19\pm0.06}{\volt}
\end{equation}
Wobei $U_B = \SI{11}{\volt}$ und ${U_B}_\text{eff} = \SI{9.813\pm0.063}{\volt}$, die Stelle des Maximums mit $\symup\Delta U_A$ als Unsicherheit.
\\
Die Messwerte für die Energieverteilung bei $\SI{140}{\degreeCelsius}$ sind in Tabelle \ref{tab:e140} aufgetragen.
Auch hier ist die Höhe proportional zu $I_A$.
\begin{table}[H]
    \caption{Messwerte der Energiekurve bei $\SI{140}{\degreeCelsius}$.}
    \label{tab:e140}
    \centering
    \begin{tabular}{S[table-format=2.1(0)e0] S[table-format=2.2(0)e0]}
        \toprule
{$U/\si{\volt}$} & { Höhe$/\si{\centi\meter}$} \\
		\midrule
0.0  & 14.60 \\
1.0  & 12.20 \\
2.0  & 9.45 \\
3.0  & 6.60 \\
4.0  & 3.85 \\
4.5  & 2.65 \\
5.0  & 1.90 \\
6.0  & 1.10 \\
7.0  & 1.30 \\
8.0  & 1.05 \\
9.0  & 0.60 \\
10.0 & 0.30 \\
       \bottomrule
    \end{tabular}
\end{table}

\noindent Die differentielle Energieverteilung ist in Diagramm \ref{fig:e140} zu sehen.

\begin{figure}[H]
	\centering
	\includegraphics[width=0.8\textwidth]{build/energie140.pdf}
	\caption{Messwerte der Energiekurve bei $\SI{140}{\degreeCelsius}$.}
	\label{fig:e140}
\end{figure}
\noindent

\subsection{Bestimmung der Anregungsenergie}
Um die Anregungsenergie des Qecksilbers zu bestimmen wird die Franck-Hertz-Kurve betrachtet, insbesondere der Abstand zwischen ihren lokalen Maxima, soweit diese noch erkennbar sind.
Die Abstände der Maxima sind in Tabelle \ref{tab:fh} aufgetragen.
\begin{table}[H]
    \caption{Messwerte der Energiekurve bei $\SI{140}{\degreeCelsius}$.}
    \label{tab:fh}
    \centering
    \begin{tabular}{S[table-format=1.2(0)e0]}
        \toprule
{$U_1/\si{\volt}$} \\
		\midrule
5.54 \\
4.57 \\
4.67 \\
4.78 \\
4.67 \\
4.89 \\
5.11 \\
       \bottomrule
    \end{tabular}
\end{table}

\noindent Für die Spannung $U_1$ ergibt sich ein Mittelwert von:
\begin{equation}
	\overline{U_1} = \SI{4.89\pm0.13}{\volt}
\end{equation}
Es ergibt sich somit eine Abstrahlungsfrequenz des Quecksilbers von
\begin{equation}
	\nu = \frac{e}{h}\, \overline{U_1} = \SI{1.18\pm0.3e15}{\hertz},
\end{equation}
wobei $h$\cite{plank} das Plank'sche Wirkungsquantum ist.

\subsection{Bestimmung der Ionisierungsenergie}
Zur Bestimmung der Ionisationsenergie wird aus dem zugehörigen Graphen der Bereich um die positiv steilste Stelle mittels linearer Regression gefittet.
Am Schnittpunkt der erhaltenen Gerade mit der $x$-Achse lässt sich $U_\text{ion} + K$ ablesen.
Die aus dem Graphen abgelesenen Messwerte sind in Tabelle \ref{tab:ion} aufgetragen.
Die Höhe ist wieder proportional zu $I_A$.
\begin{table}[H]
    \caption{Messwerte der Ionisationsenergie.}
    \label{tab:ion}
    \centering
    \begin{tabular}{S[table-format=2.2(0)e0] S[table-format=1.1(0)e0]}
        \toprule
{$U/\si{\volt}$} & { Höhe$/\si{\centi\meter}$} \\
		\midrule
25.00  & 2.1 \\
26.09  & 2.3 \\
27.17  & 2.6 \\
28.26  & 3.0 \\
29.35  & 3.3 \\
30.22  & 3.6 \\
       \bottomrule
    \end{tabular}
\end{table}
\noindent Der lineare Fit wird mit Python/SciPy mit der Vorschrift
\begin{equation}
	f(x) = mx+n
\end{equation}
erstellt und ist in Abbildung \ref{fig:ion} dargestellt.

\begin{figure}[H]
	\centering
	\includegraphics[width=0.8\textwidth]{build/ion.pdf}
	\caption{Messwerte und linearer Fit der Ionisationsenergiemessung.}
	\label{fig:ion}
\end{figure}
\noindent Die Parameter ergeben sich zu:
\begin{align}
	m &= \SI{0.29\pm0.01}{\centi\meter\per\volt} \\
	n &= \SI{-5.33\pm0.36}{\centi\meter}
\end{align}
Die Ionisationsspannung ergibt sich zu:
\begin{equation}
	U_\text{ion} = \frac{-n}{m} - K = \SI{17.19\pm1.40}{\volt} 
\end{equation}
Die Unsicherheit berechnet sich wie folgt:
\begin{equation}
	\symup\Delta U_\text{ion} = \sqrt{\left(\frac{\symup\Delta n}{m}\right)^2 + \left(\frac{n\symup\Delta m}{m^2}\right)^2 + \symup\Delta K ^2} 
\end{equation}
