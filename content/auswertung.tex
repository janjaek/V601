\section{Auswertung}
\label{sec:Auswertung}
\subsection{Bestimmung der differentiellen Energieverteilung der Elektronen}
Die Messwerte zur Bestimmung der differentiellen Energieverteilung bei $\SI{20}{\degreeCelsius}$ sind in Tabelle \ref{tab:e20} aufgetragen.
Die Messwerte werden dem Graphen des $XY$-Schreibers so entnommen, dass für große Steigung viele Werte aufgenommen werden.
Die Höhe ist hierbei eine Größe proportional zum Strom $I_A$.
Um die Eichung der $X$-Achse des $XY$-Schreibers zu ermitteln wird der Abstand der eingezeichneten Markierungen gemessen und anschließend gemittelt.
Die Werte der Eichung sind in Tabelle \ref{tab:eich1} aufgetragen.

\begin{table}[H]
    \caption{Abstände der Eichmarkierungen.}
    \label{tab:eich1}
    \centering
    \begin{tabular}{S[table-format=1.2(0)e0]}
        \toprule
{Abstand der Markierungen$/\si{\centi\meter}$} \\
		\midrule
2.2 \\
2.4 \\
2.35 \\
2.35 \\
2.2 \\
2.35 \\
2.4 \\
2.35 \\
2.42 \\
2.55 \\
        \bottomrule
    \end{tabular}
\end{table}
\noindent
Es ergibt sich somit, bei einer Einteilung von $\SI{1}{\volt}$ pro Einheit, ein mittlerer Umrechnungsfaktor $A_1$ von:
\begin{equation}
	\overline{A_1} = \SI{2.36\pm0.03}{\centi\meter\per\volt}
\end{equation}

\begin{table}[H]
    \caption{Messwerte der integralen Energiekurve bei $\SI{20}{\degreeCelsius}$.}
    \label{tab:e20}
    \centering
    \begin{tabular}{S[table-format=2.3(5)e0] S[table-format=2.2(0)e0]}
        \toprule
{$U/\si{\volt}$} & { Höhe$/\si{\centi\meter}$} \\
		\midrule
1.000\pm0.013  &      14.65 \\
2.000\pm0.025  &      14.60 \\
3.000\pm0.038  &      14.50 \\
4.000\pm0.051  &      14.40 \\
5.000\pm0.064  &      14.20 \\
6.000\pm0.076  &      14.00 \\
6.500\pm0.083  &    13.85 \\
7.000\pm0.089  &      13.60 \\
7.500\pm0.095  &    13.30 \\
8.000\pm0.102  &      12.95 \\
8.500\pm0.108  &    12.40 \\
9.000\pm0.114  &      11.55 \\
9.250\pm0.118  &   10.95 \\
9.500\pm0.121  &    10.00 \\
9.625\pm0.122  &  8.70 \\
9.750\pm0.124  &   6.30 \\
9.813\pm0.125  &  4.5 \\
9.875\pm0.126  &  2.70 \\
10.000\pm0.127  &      1.00 \\
10.125\pm0.129  &	 0.50 \\
10.250 \pm0.130  &	 0.40 \\
        \bottomrule
    \end{tabular}
\end{table}

\noindent Die differentielle Energieverteilung, welche sich ergibt aus $I_A(U_A)-I_A(U_A+\symup\Delta U_A)$, aufgetragen gegen $U_A$, ist in Abbildung \ref{fig:e20} dargestellt.
\begin{figure}[H]
	\centering
	\includegraphics[width=0.8\textwidth]{build/energie20.pdf}
	\caption{Differentielle Energieverteilung der Elektronen bei $\SI{20}{\degreeCelsius}$.}
	\label{fig:e20}
\end{figure}

\noindent Aus der Abbildung \ref{fig:e20} lässt sich nun gemäß Gleichung \eqref{eq:k} über das Maximum das Kontaktpotential bestimmen:
\begin{equation}
	K = U_B - {U_B}_\text{eff} = \SI{1.19}{\volt}
\end{equation}
Wobei $U_B = \SI{11}{\volt}$ und ${U_B}_\text{eff} = \SI{9.813\pm0.063}{\volt}$, die Stelle des Maximums sind.
\\
Die Messwerte für die Energieverteilung bei $\SI{140}{\degreeCelsius}$ sind in Tabelle \ref{tab:e140} aufgetragen.
Auch hier ist die Höhe proportional zu $I_A$.
\begin{table}[H]
    \caption{Messwerte der integralen Energiekurve bei $\SI{140}{\degreeCelsius}$.}
    \label{tab:e140}
    \centering
    \begin{tabular}{S[table-format=2.1(3)e0] S[table-format=2.2(0)e0]}
        \toprule
{$U/\si{\volt}$} & { Höhe$/\si{\centi\meter}$} \\
		\midrule
0.0\pm0.0   & 14.60 \\
1.0\pm0.0   & 12.20 \\
2.0\pm0.0   & 9.45 \\
3.0\pm0.0   & 6.60 \\
4.0\pm0.1   & 3.85 \\
4.5\pm0.1   & 2.65 \\
5.0\pm0.1   & 1.90 \\
6.0\pm0.1   & 1.10 \\
7.0\pm0.1   & 1.30 \\
8.0\pm0.1   & 1.05 \\
9.0\pm0.1   & 0.60 \\
10.0\pm0.1  & 0.30 \\
       \bottomrule
    \end{tabular}
\end{table}

\noindent Es wird für diese Werte der gleiche Umrechnungsfaktor $A_1$, wie zuvor benutzt.
Die differentielle Energieverteilung ist in Abbildung \ref{fig:e140} zu sehen.

\begin{figure}[H]
	\centering
	\includegraphics[width=0.8\textwidth]{build/energie140.pdf}
	\caption{Differentielle Energieverteilung bei $\SI{140}{\degreeCelsius}$.}
	\label{fig:e140}
\end{figure}
\noindent

\subsection{Bestimmung der Anregungsenergie}
Um die Anregungsenergie des Quecksilbers zu bestimmen wird die Franck-Hertz-Kurve betrachtet, insbesondere der Abstand zwischen ihren lokalen Maxima, soweit diese noch erkennbar sind.
In Tabelle \ref{tab:eich2} sind zunächst die Abstände der Eichmarkierungen aufgetragen.
\begin{table}[H]
    \caption{Abstände der Eichmarkierungen.}
    \label{tab:eich2}
    \centering
    \begin{tabular}{S[table-format=1.2(0)e0]}
        \toprule
{Abstand der Markierungen$/\si{\centi\meter}$} \\
		\midrule
2.3 \\
2.1 \\
2.35 \\
2.3 \\
2.35 \\
2.2 \\
2.25 \\
2.25 \\
2.45 \\
        \bottomrule
    \end{tabular}
\end{table}
\noindent
Es ergibt sich somit, bei einer Einteilung von $\SI{5}{\volt}$ pro Einheit, ein mittlerer Umrechnungsfaktor $A_2$ von:
\begin{equation}
	\overline{A_2} = \SI{0.46\pm0.01}{\centi\meter\per\volt}
\end{equation}

Die Abstände der Maxima sind in Tabelle \ref{tab:fh} aufgetragen.
\begin{table}[H]
    \caption{Messwerte der Energiekurve bei $\SI{140}{\degreeCelsius}$.}
    \label{tab:fh}
    \centering
    \begin{tabular}{S[table-format=1.2(4)e0]}
        \toprule
{$U_1/\si{\volt}$} \\
		\midrule
5.54\pm0.06 \\
4.57\pm0.05 \\
4.67\pm0.05 \\
4.78\pm0.05 \\
4.67\pm0.05 \\
4.89\pm0.05 \\
5.11\pm0.05 \\
       \bottomrule
    \end{tabular}
\end{table}

\noindent Für die Spannung $U_1$ ergibt sich ein Mittelwert von:
\begin{equation}
	\overline{U_1} = \SI{4.89\pm0.13}{\volt}
\end{equation}
Es ergibt sich somit eine Abstrahlungsfrequenz des Quecksilbers von
\begin{equation}
	\nu = \frac{e}{h}\, \overline{U_1} = \SI{1.18\pm0.3e15}{\hertz},
\end{equation}
wobei $h$\cite{plank} das Plank'sche Wirkungsquantum ist.

\subsection{Bestimmung der Ionisierungsenergie}
Zur Bestimmung der Ionisationsenergie wird aus dem zugehörigen Graphen der Bereich um die positiv steilste Stelle mittels linearer Regression gefittet.
Am Schnittpunkt der erhaltenen Gerade mit der $x$-Achse lässt sich $U_\text{ion} + K$ ablesen.
In Tabelle \ref{tab:eich3} sind zunächst die Abstände der Eichmarkierungen aufgetragen.
\begin{table}[H]
    \caption{Abstände der Eichmarkierungen.}
    \label{tab:eich3}
    \centering
    \begin{tabular}{S[table-format=1.2(0)e0]}
        \toprule
{Abstand der Markierungen$/\si{\centi\meter}$} \\
		\midrule
2.25 \\
2.4 \\
2.2 \\
2.4 \\
2.25 \\
2.25 \\
2.45 \\
2.4 \\
2.15 \\
        \bottomrule
    \end{tabular}
\end{table}
\noindent
Es ergibt sich somit, bei einer Einteilung von $\SI{5}{\volt}$ pro Einheit, ein mittlerer Umrechnungsfaktor $A_2$ von:
\begin{equation}
	\overline{A_2} = \SI{0.46\pm0.01}{\centi\meter\per\volt}
\end{equation}

Die aus dem Graphen abgelesenen Messwerte sind in Tabelle \ref{tab:ion} aufgetragen.
Die Höhe ist wieder proportional zu $I_A$.
\begin{table}[H]
    \caption{Messwerte der Ionisationsenergie.}
    \label{tab:ion}
    \centering
    \begin{tabular}{S[table-format=2.2(4)e0] S[table-format=1.1(0)e0]}
        \toprule
{$U/\si{\volt}$} & { Höhe$/\si{\centi\meter}$} \\
		\midrule
25.00\pm0.25   & 2.1 \\
26.09\pm0.26   & 2.3 \\
27.17\pm0.27   & 2.6 \\
28.26\pm0.28   & 3.0 \\
29.35\pm0.29   & 3.3 \\
30.22\pm0.30   & 3.6 \\
       \bottomrule
    \end{tabular}
\end{table}
\noindent Der lineare Fit wird mit Python/SciPy\cite{scipy} mit der Vorschrift
\begin{equation}
	f(x) = mx+n
\end{equation}
erstellt und ist in Abbildung \ref{fig:ion} dargestellt.

\begin{figure}[H]
	\centering
	\includegraphics[width=0.8\textwidth]{build/ion.pdf}
	\caption{Messwerte und linearer Fit der Ionisationsenergiemessung.}
	\label{fig:ion}
\end{figure}
\noindent Die Parameter ergeben sich zu:
\begin{align}
	m &= \SI{0.29\pm0.01}{\centi\meter\per\volt} \\
	n &= \SI{-5.33\pm0.36}{\centi\meter}
\end{align}
Die Ionisationsspannung ergibt sich zu:
\begin{equation}
	U_\text{ion} = \frac{-n}{m} - K = \SI{17.19\pm1.40}{\volt} 
\end{equation}
Die Unsicherheit berechnet sich wie folgt:
\begin{equation}
	\symup\Delta U_\text{ion} = \sqrt{\left(\frac{\symup\Delta n}{m}\right)^2 + \left(\frac{n\symup\Delta m}{m^2}\right)^2 + \symup\Delta K ^2} 
\end{equation}
